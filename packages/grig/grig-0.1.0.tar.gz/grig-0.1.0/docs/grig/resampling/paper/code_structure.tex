\subsection{Language and packages}
\label{subsec:packages}

The resampling code is written in Python \footnote{Available from the
\href{http://www.python.org}{Python Software Foundation}}
(see~\citet{Python1995}) version 3.
Python is a dynamic open source programming language supported by a vast
ecosystem of specialized packages of which we rely on only a few:
Data such as the sample values, errors, $\Phi$, etc., are stored as Numpy arrays
(\citet{Numpy}) and processed using JIT (Just-In-Time) compiled Numba
functions (\citet{Numba}).
Numba was instrumental in performing multidimensional resampling in a
reasonable time frame since Python is notoriously slow when compared to
other languages, and allows for operations to be performed on a time scale
comparable with the C language.
The balltree algorithm (\citet{Balltree}) from the Scikit-learn package
(\citet{sklearn}) is used to quickly determine which samples are inside the
window region of a resampling point (see
figure~\ref{fig:single_thread_reduction}).
Finally, parallel processing is handled by the Joblib
\footnote{Available from \href{https://joblib.readthedocs.io/en/latest/}
{https://joblib.readthedocs.io/en/latest/}}.
The resampling code exists as a submodule of the toolkit module in the SOFIA
data reduction pipeline package (REDUX) that is not yet publicly available.
However, efforts are being made to release the REDUX package within the year.

\subsection{Classes and Structure}
\label{subsec:class-structure}

Figure~\ref{fig:software-flow} gives a high level representation of the
various classes and process flow.
The first step is to create an instance of the Resampler class, initialized
with some parameters that must include the window defining the
''local'' region ($\omega$), the polynomial fit order, and the data we wish to
resample.
The user may interact with the Resampler through some type of interface.
Nearly all examples in this paper were generated from the Python command line
and Jupyter notebooks, although some FIFI-LS images were created using the
SOFIA data reduction pipeline (REDUX).

During the instantiation of the Resampler object, the sample coordinates ($X$)
are used to initialize a Tree object that allocates samples into blocks and
their associated neighborhoods as described in
section~\ref{subsec:search-problem} as a function of the window region $\omega$.
The $\Phi(x)$ terms (see
section~\ref{subsec:polynomial-representation-of-k-dimensional-data}) are also
created at this stage and stored as a Tree attribute.

Once the Resampler object has been created, a reduction may be started by
supplying a set of reduction parameters (such as weighting schemes) and a set
of coordinates (resampling points) containing the desired locations for the
fitted output values.
Another Tree object is created by the reduction manager from the resampling
point coordinates ($V$), allocating blocks and creating a set of $\Phi(V)$
terms.

Each intersection of a reduction Tree block with a sample Tree neighborhood is
than passed in parallel from the reduction manager to the main engine of the
resampling algorithm (the box labelled ``Fit'' in
figure~\ref{fig:software-flow}).
For a single block of resampling points, the sample sets $\Omega$ for each
point are derived simultaneously from the sample neighborhood using the
balltree algorithm.
A fit at each point is processed in series before being passed back to the
reduction manager where they are aggregated.
Once all parallel reductions are complete, the results become available to the
Resampler where they are either used to update the reduction parameters when
determining an adaptive solution (resulting in a second reduction) or passed
back to the user.

\begin{figure}[H]
  \begin{center}
  \includegraphics[width=\linewidth]{images/software_flow.png}
  \caption{Main software objects and process flow for the full resampling
           algorithm.  Data and variables are denoted by a green open circle.
           Notes and actions are marked by a filled black circle.  Aggregate
           objects (can exist independently of the parent object) are marked
           with open diamond heads, while composite objects (cannot exist
           without parent object) are marked with a filled diamond head.}
  \label{fig:software-flow}
  \end{center}
\end{figure}


