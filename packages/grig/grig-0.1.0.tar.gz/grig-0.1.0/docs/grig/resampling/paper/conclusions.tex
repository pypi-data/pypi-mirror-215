\section{Future Work}\label{sec:future-work}

This paper is primarily concerned with local polynomial regression, but could
easily be extended to other families of functions or filters.
For example, Gaussian, Lorentzian, Voigt, trigonometric, median etc.
Modified versions of the algorithm already exist for the weighted median
or mean of a local region, but do not yet implement adaptive weighting.

Another improvement that could be made would be to allow for variable window
regions.
Currently, the user must explicitly select a single set of window dimensions
($\omega$) applied around each resampling point.
In practice, this should be set to some minimum value to allow sufficient
samples from which to derive a fit, but also be large enough to account for
expansion of the adaptive weighting kernel.
Since computational complexity is $\propto \mathcal{O}(|\Omega|)$, the window
dimensions can have a significant effect on total reduction time.
Given that selecting a small window region can invalidate certain fits, the
user may be forced to choose a large window region that rectifies this
situation effecting only a few points at the cost of increased processing time.

\section{Conclusion}\label{sec:conclusion}
We have introduced a new procedure by which simultaneous multi-dimensional
interpolation can be carried out.
Consideration has been given to the distribution of irregularly sampled data,
and with appropriate choices, spurious fits can be reduced or eliminated.
An optional method is available to determine adaptive weighting factors in a
single step, approximating $\chi_r^2 \to 1$ on subsequent reductions.
These methods have been applied to resample three-dimensional irregular data
observed with the FIFI-LS instrument onto a regular data cube.
